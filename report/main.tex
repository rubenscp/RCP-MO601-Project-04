% Autor: Joel S. Evangelista             FEQ/Unicamp (016)

\documentclass[a4paper, 12pt, fleqn]{article}

\usepackage[portuges]{babel}
\usepackage[utf8]{inputenc}
\usepackage[a4paper,left=2cm,right=2cm,top=2cm,bottom=2cm]{geometry}
\usepackage{amsmath}            %equações
\usepackage{array}
\usepackage{breakcites}         %quebra link nas referências de sites
\usepackage[labelfont=bf, labelsep=endash]{caption}  %caption config
\usepackage{cite}               %referências
\usepackage{color}              %texto colorido
\usepackage{colortbl}           %cor nas tabelas
\usepackage{float}
%\usepackage{fourier}
\usepackage{graphicx}           %Figuras lado a lado
\usepackage{hyperref}
\usepackage{indentfirst}
\usepackage{longtable}          %tabelas longas
\usepackage{makecell}
\usepackage[version=4]{mhchem}  %equações químicas
\usepackage{multicol}           %colunas
\usepackage{nomencl}            %nomenclatura
\usepackage{setspace}           %espaçamento entre linhas
\usepackage{subfig}
\usepackage{subfiles}
\usepackage{tabularx}
\usepackage{tikz}
\usepackage{titlesec}
\usepackage{xfrac}
\usepackage[alf]{abntex2cite}   %estilo de referências
% \usepackage{algorithm}
\usepackage{algpseudocode}
% \usepackage[portuguese, ruled, linesnumbered]{algorithm2e}
\usepackage[ruled, vlined, linesnumbered]{algorithm2e}

% incluido por Rubens em 01/05/2023
%\usepackage{cite}
%\usepackage{biblatex}
%\addbibresource{Tarefas/reference.bib}  

\makenomenclature               %nomenclatura
% \setlength{\parindent}{1cm}     %indentação do parágrafo em 1
\setlength{\parindent}{0.5cm}     %indentação do parágrafo em 1

% Configurações da seção, subseção e subsubseção
\titleformat{\section}
{\normalfont\large\bfseries}{\thesection}{1em}{}
\titleformat{\subsection}
{\normalfont\large\bfseries\itshape}{\thesubsection}{1em}{}
\titleformat{\subsubsection}
{\normalfont\large\bfseries\itshape}{\thesubsubsection}{1em}{}

%Configurações da nomenclatura
\newcommand{\nomunit}[1]{
\renewcommand{\nomentryend}{\hspace*{\fill}#1}}

% Configurações nas referências
\hypersetup{
	colorlinks=true,
	linkcolor=black,
	urlcolor=black,
	linktoc=all,
	urlcolor=black,
    citecolor=black,
    linkcolor=black
}

\begin{document}
	
	\fontfamily{ptm} \selectfont    %fonte do texto Times
	\onehalfspacing                 %espaçamento entre linhas de 1,5
	
	\subfile{Tarefas/capa}
	
	\renewcommand{\contentsname}{Índice} %renomeação do índice
	\tableofcontents                    \newpage
	
	\renewcommand{\nomname}{Nomenclatura} %renomeio da nomenclatura
	\printnomenclature                  \newpage
	\subfile{Tarefas/nomenclatura}	

	\subfile{Tarefas/introducao.tex}
	\subfile{Tarefas/descricao_geral}
	\subfile{Tarefas/ambiente_desenvolvimentoral}
	\subfile{Tarefas/algoritmo_simulacao}
	\subfile{Tarefas/testes}
	\subfile{Tarefas/consideracoes_aprendizado.tex}
	
	% \let\Section\section\def\section*#1{\Section{#1}}
	% \renewcommand{\refname}{Referências Bibliográficas}
	% \bibliography{Tarefas/reference.bib}

	\begin{thebibliography}{99}
		\bibitem{riscv_isa} Andrew Waterman1, and Krste Asanovi, The RISC-V Instruction Set Manual, Volume I: User-Level ISA, Document Version 2.2, 2017. https://riscv.org/wp-content/uploads/2017/05/riscv-spec-v2.2.pdf.
		\bibitem{riscv_iss} Msyksphinz. RISC-V Instruction Set Specifications, 2019. Acessado em abril de 2023. Disponível em https://msyksphinz-self.github.io/riscv-isadoc/html/rvi.html.
		\bibitem{riscv_iss} RISC-V Instruction Encoder/Decoder. Acessado em abril de 2023. Disponível em https://luplab.gitlab.io/rvcodecjs
	\end{thebibliography}

\end{document}
